
\section{Abstract and Interface}
\subsection{Abstract Classes and Extends}

\begin{frame}{Extends}
\begin{itemize}
\item When declaring a class, we can say it \emph{extends} another class. \pause
\item If a class $A$ extends another class $B$, we say that class $A$ is a subclass of $B$.  \pause
\item Another way to say this is that $A$ is $B$'s child, and $B$ is $A$'s parent.  \pause
\item We can also say that $A$ is derived from $B$, inheriting $B$'s methods.  \pause
\item A subclass can use any of the methods of its parent class.  \pause
\item A class can only extend one other class. However, a class can extend a class that extends another class that... etc.
\item To access the methods of a parent class, use the keyword \texttt{\code{super}}.
\end{itemize}
\end{frame}


\begin{frame}{Overriding}
\begin{itemize}
\item If a subclass declares a method with the same modifiers, name, and arguments as a parent, we say that the subclass \emph{overrides} the method. \pause
\item \code{\texttt{@Override}} declares that a method overrides a parent's method. If the method does not override a parent method this will create an error.  \pause
\item Take for example the person class with a method work. We then might have two subclasses, a teacher and a student, that extend person. \pause
\end{itemize}
\end{frame}

\begin{frame}{Overriding}
\begin{itemize}

\item Both the student and the teacher are persons. Thus they can do anything other persons can do.  \pause
\item When a teacher works, she or he creates homework. When a student does work, she or he finishes homework.  \pause
\item All classes \emph{descend} from the \code{Object} class. This means all objects have inherited the methods from the \code{Object} class. Furthermore, in a way \code{null} descends from all classes, though it does not contain any methods.
\end{itemize}
\end{frame}


\begin{frame}[fragile]{Different Persons}
\begin{semiverbatim}\code{\small
public class Person\{
    public void work()\{\}
\}

public class Teacher extends Person\{
    public void work()\{
    //Create Homework Here
    \}
\}

public class Student  extends Person\{
    public void work()\{
    //Finish Homework Here
    \}
\}
}\end{semiverbatim}
\end{frame}

\begin{frame}{Why Bother?}
\begin{center}
\includegraphics[width = 4.5cm]{winnie}

\emph{Using classes, subclasses, and hierarchies of objects simplifies our code by making it 1. shorter and 2. organizing it by type.}\end{center}
\end{frame}

\begin{frame}{Abstract Classes}
\begin{itemize}
\item Abstract classes require an object to extend them in order to be used. \pause
\item Abstract classes can have both normal methods (the ones we are used to), as well as abstract methods. Abstract methods \emph{have to} be overriden by the subclass.
\item Since the methods must be seen by subclasses in order to be overriden, they cannot be private. \pause
\item To declare an abstract method, you use the \code{abstract} keyword followed by the method declaration followed by a semicolon:

\begin{center}
\code{\texttt{abstract public void collision(SimpleObject s);}}
\end{center}

\end{itemize}
\end{frame}

\begin{frame}[fragile]{Sneak Peak \#4}
\begin{semiverbatim}\code{
public abstract class SimpleObject \{
   protected final int[] off = { 0, 0 };
   abstract public void collision(SimpleObject s);
   abstract public void update();
   abstract public char id();
   public String getInfo()\{
      return "";
   \}
   public SimpleObject getCopy(String s)\{
      return null;
   \}
}\end{semiverbatim}

...
\end{frame}


\subsection{Interfaces and Implements}
\begin{frame}{Interfaces}
\begin{itemize}
\item Like an abstract class, an \emph{interface} must be \emph{implemented} by a subclass. \pause
\item Unlike an abstract class, interfaces \emph{only} contains empty methods that must be overriden, and they only contain static final variables. \pause
\item The keyword \emph{final} makes a variable constant. Final variables cannot change \pause
\item However, interfaces are useful since you can \emph{implement} more than one.
\item When we access objects through shared methods of a parent class or an interface, we call that (inclusion) \emph{polymorphism}.
\end{itemize}
\end{frame}

\begin{frame}[fragile]{The KeyListener Class}
\begin{semiverbatim}\code{
public interface KeyListener extends EventListener \{

    public void keyTyped(KeyEvent e);

    public void keyPressed(KeyEvent e);

    public void keyReleased(KeyEvent e);
\}
}\end{semiverbatim}
\begin{center}
Interfaces can extend (not implement) other interfaces. Interfaces cannot extend classes.
\end{center}
\end{frame}

\begin{frame}[fragile]{Find the Abstract Class Errors!}
\begin{semiverbatim}\code{\small
public abstract class animal \{
    abstract int population = 0;

    private abstract void eat(food f);
    public abstract static void sleep();
    protected int getPopulation();
    public abstract void makeNoise(int noise);

    private void sayHello()\{
       System.out.println("I am an animal!");
    \}
    public abstract void move()\{
       //Move code here...
    \}
\}
}\end{semiverbatim}

\end{frame}

\begin{frame}[fragile]{Find the Interface Errors!}
\begin{semiverbatim}\code{
public interface updateListener implements updater \{

    public final FASTUPDATE = 5;
    String s;

    public abstract void updateEvent1(int updateID);
    private void autoUpdate();
    public voidupdateEvent2(int updateID);
\}
}\end{semiverbatim}\pause

\begin{center}If we fixed all the errors in the above code, what methods would a class that implements updateListener have to override?\end{center}
\end{frame}

\section{Bitwise Operations}
\subsection{Bitwise Operations}
\begin{frame}{Introduction}
\begin{itemize}
\item So far we know the following operators for integers: \code{\texttt{+}}, \code{\texttt{-}}, \code{\texttt{\textsuperscript{*}}}, \code{\texttt{/}}, \code{\texttt{\%}}, and \code{\texttt{=}}.
\item In addition to these there are bitwise operators: \code{\texttt{\^}}, \code{\texttt{\&}}, \code{\texttt{|}}, \code{\texttt{\textasciitilde}}, \code{\texttt{>>}}, \code{\texttt{>>>}}, and \code{\texttt{<<}}.
\item Finally there is one \emph{ternary} operator \code{\texttt{?:}}. It is shorthand for an if else statement.
\end{itemize}
\end{frame}

\begin{frame}[fragile]{XOR and AND}
\begin{itemize}
\item Exclusive OR or XOR(\code{\texttt{\^}}) outputs a `1' if one bit is a `1' and the other is a `0':

\begin{semiverbatim}\code{
11001101
10010100
01011001}\end{semiverbatim}
\pause
\item AND(\code{\texttt{\&}}) outputs a `1' if both bits are a `1':

\begin{semiverbatim}\code{
01100110
10101010
00100010}\end{semiverbatim}
\end{itemize}
\end{frame}


\begin{frame}[fragile]{OR and NOT}
\begin{itemize}
\item OR(\code{\texttt{|}}) outputs a `1' if at least one bit is a `1':

\begin{semiverbatim}\code{
10001101
10100100
10101101}\end{semiverbatim}
\pause
\item NOT(\code{\texttt{\textasciitilde}}) ``flips" the bits:

\begin{semiverbatim}\code{
10011010
01100101}\end{semiverbatim}
\end{itemize}
\end{frame}


\begin{frame}[fragile]{Bit Shift}
\begin{itemize}
\item Shift-left(\code{\texttt{<<}}) moves all bits to the left padding with a `0': (shifting by 1)

\begin{semiverbatim}\code{
10011011
00110110}\end{semiverbatim}
\pause

\item Logical Shift-right(\code{\texttt{>>}}) and Arithmetic Shift-right(\code{\texttt{>>>}}) moves all bits to the right. Logical shift right pads with a `0' while arithmetic shift copies the leading bit: (shifting by 1)

\begin{semiverbatim}\code{
01001101
00101010}\end{semiverbatim}
\end{itemize}
\end{frame}

\begin{frame}[fragile]{Playing with Bitwise Operators}
\begin{semiverbatim}\code{\scriptsize
public class Bits \{

    public static int a = 5;
    public static int b = 27;
    public static int c = 17;
    public static int d = 14;

    public static void calculate()\{
        int e = b \& c;
        int f = b | c;
        int g = b >> 1;
        int h = d \& (\textasciitilde c);
        int i = a \^ b;
        int j = b << 3;
        int k = b \& ((c | d) >> 2);
        int l = (({\textasciitilde}a) \& c) \^ (({\textasciitilde}b) \& d);
        int m = (a << (a - 1)) \^ b;
    \}
\}
}\end{semiverbatim}

\only<1>{What is \code{a} in binary?}
\only<2>{What is \code{b} in binary?}
\only<3>{What is \code{c} in binary?}
\only<4>{What is \code{d} in binary?}
\only<5>{What is \code{e} in binary?}
\only<6>{What is \code{f} in binary?}
\only<7>{What is \code{g} in binary?}
\only<8>{What is \code{h} in binary?}
\only<9>{What is \code{i} in binary?}
\only<10>{What is \code{j} in binary?}
\only<11>{What is \code{k} in binary?}
\only<12>{What is \code{l} in binary?}
\only<13>{What is \code{m} in binary?}
\end{frame}


\begin{frame}{Last Chance for Points}
Define the following keywords:

\begin{center}
\code{public\\ \pause
final\\ \pause
static\\ \pause
private\\ \pause
protected\\ \pause
void\\ \pause
abstract\\ \pause
class\\ \pause
interface\\ \pause
implements\\ \pause
extends
}\end{center}
\end{frame}
