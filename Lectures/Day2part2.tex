
\section{Making Methods!}

\subsection{Modifiers}

\begin{frame}[fragile]{WarmUp}
\begin{semiverbatim}\code{\scriptsize
import java.util.Scanner;

public class quizGame\{
   static String[] questions = \{"Pick a number (1-4)",
   "Pick a color (1. Red, 2. Blue, 3. Green, 4. Purple)",
   "Pick an animal (1. Panda, 2. Squirrel, 3. Whale, 4. Jigglypuff) "\};

   public static void main(String[] args)\{
      int i = 0;
      Scanner input = new Scanner(System.in);
      for(String s : questions)\{
         System.out.println(s);
         i = (i + input.nextInt())\%4;
      \}
}\end{semiverbatim}
\end{frame}


\begin{frame}[fragile]{WarmUp}
\begin{semiverbatim}\code{\scriptsize
switch (i)\{
   case 0 :
   System.out.println("You will have a super lucky day!");
   break;
   case 1 :
   System.out.println("You will learn a lot today!");
   break;
   case 2 :System.out.println("You will have a bad day today.");
   break;
   case 3 :System.out.println("Today is a good day to make friends");
   break;
   default:
   System.out.println("Error!");
\}
input.close();
\}
\}
}\end{semiverbatim}
\end{frame}

\begin{frame}[fragile]{A simple method}
\begin{itemize}
\item Believe it or not, we know enough syntax to make any program conceivable (though we do not know other things like how to interface to the screen). \pause
\item However, if we put everything in the main method, our code will be extremely long and unreadable!
\item We need methods:
\begin{center}\texttt{\code{public static <returnType> <methodName>(<OptionalArguments>)\{ ... \}}}\end{center}
\item A \emph{static} method can take in \emph{arguments}, return a value, and change the state of other static variables. Lets try it:
\begin{semiverbatim}\code{
public static int RollXd20(int number)\{\scriptsize
   return (int)(number\textsuperscript{*}20\textsuperscript{*}math.Random()) +1;
\}
}\end{semiverbatim}
\end{itemize}
\end{frame}

\begin{frame}{Static vs. Non-Static}
\begin{itemize}
\item For the time being we are going to keep everything \emph{public}.
\item However, we can start using \emph{non-static}, instance variables and methods. \pause
\item So far we've used a few classes that require \emph{instantiation} (e.g. Scanner, arrays) \pause
\item \emph{Static} methods and variables do not require instantiating a class.
\item Non static methods do require instantiating an object, and each instance gets its own copy of all the non-static variables.
\end{itemize}
\end{frame}

\begin{frame}[fragile]{Creating an Object}
\begin{semiverbatim}\code{
public class Person\{
    static int population = 0;
    int counter = 0;
    int ssn;

    public Person(int ssn)\{
       this.ssn = ssn;
       population++;
       counter ++;
    \}

    public in getSSN()\{
       return ssn;
    \}
\}
}\end{semiverbatim}
\end{frame}

\begin{frame}[fragile]{Instantiating an Object}
\begin{semiverbatim}\code{
public class PersonTest\{
    public static void main(String[] args)\{
        Person p = new person();
        p.getSSN();
    \}
\}
}\end{semiverbatim}
\end{frame}

\begin{frame}{Case study: Objects with a Static Variable}
\begin{center}
Create a static method in the Person class that returns the population variable. Next in the PersonTest class, create more than one person using a loop. How does the population change? Finally, do the same for the counter variable. Does it change?
\end{center}
\end{frame}

\begin{frame}{Refactor your 2D RPG into methods}
\begin{center}
Your RPG from before lunch can be augmented with methods. For example, you can put all of the logic for the monster in a monster method. Try to put all of the logic for the monster in its own method, then recompile your program and make sure it works.\\
\end{center}
\end{frame}

\subsection{Packages and Classes}
\begin{frame}{Getting organized}
\begin{center}
As you get more classes, you may want to start sorting them into different packages. Creating a package is easy. In Eclipse click \code{File$\rightarrow$New$\rightarrow$Package}. Name your package and click \code{Finish}. Don't forget, it you're using code that's in a different package, you need to use \emph{import}.
\end{center}
\end{frame}

\begin{frame}{Extra Challenge}
\begin{center}
Move your methods into different classes and or different packages. For example it might be nice to have a dice package that calculates random dice numbers for 4d20, d6, etc. After this, if your RPG is complete enough show it to me or a friend!
\end{center}
\end{frame}

\section{Recursion}

\subsection{What it is}
\begin{frame}{A Weird Way to Loop}
\begin{itemize}
\item Loops let us repeat the same code more than once, but they have one limitation: complexity. \pause
\item Sometimes code written using loops can be verbose and confusing, and for this reason programmers sometimes opt to use recursion.
\item What is a recursive function? \pause
\item Definition: $f(x) = g(f(x-1), f(x-2), ...)$. In other words if we have some output $f_{n+1}$ it depends on its previous output(s) $f_{n}$.\pause
\item Example: $f_{n+1} = !f_{n+1}$, where for all $n, f_n$ is either \texttt{\code{true}} or \texttt{\code{false}}. \pause
\item This gives us the sequence \texttt{\code{true, false, true, false, true}}...
\item What does this look like in code?
\end{itemize}
\end{frame}

\subsection{An Example}
\begin{frame}[fragile]{Recursion}
\begin{semiverbatim}\code{
class factorial\{
   public static void main(String[] args)\{
      int i = 0;
      if (args.length > 0)\{
         i = Integer.parseInt(args[0]);
      \}
      System.out.println(i + "! = " + factorial(i));
   \}
}\end{semiverbatim}
\end{frame}

\begin{frame}[fragile]{Recursion}
\begin{semiverbatim}\code{
   public static int factorial(int n)\{
      if(n > 1)\{
         return n * factorial(n-1);
      \}
      else if(n >= 0)\{
         return 1;
      \}
      return -1;
   \}
\}
}\end{semiverbatim}
\end{frame}


\begin{frame}[fragile]{Questions}
\begin{center}
What number sequence does this output?\\
$f_0, f_1, f_2, f_3 ...?$\\ \pause
Challenge: Which is faster, recursion or looping? Can you use \texttt{\code{System.nanoTime()}} to find out?
\end{center}
\end{frame}

\section{Bonus Sneak Peak and HW}
\subsection{The Game Engine}
\begin{frame}{2D Array based Collision Algorithm}
\begin{center}Download the resources file on my website into your \code{MyProjects} folder, and then extract its contents. Import the resources as a project into Eclipse. Click \code{File $\rightarrow$ Import $\rightarrow$ General $\rightarrow$ Existing Project from Workspace $\rightarrow$ Next $\rightarrow$ Browse}. Then select the \code{Resources} folder. Finally click \code{Select All $\rightarrow$ Finish}.\\

The interesting method is \code{moveCell()} in \code{SimpleSolid.java}\end{center}
\end{frame}

\begin{frame}[fragile]{Some Big If Statements}
\begin{semiverbatim}\code{
if (temp_x >= 0 && temp_x < w.map.length 
      && temp_y >= 0 && 
      temp_y < w.map[0].length) \{
   SimpleSolid s = w.map[temp_x][temp_y];
   if (s != null) \{
      s.collision(this);
      collision(s);
   \}
}\end{semiverbatim}

\end{frame}

\subsection{Practice Recursion}
\begin{frame}{Some Recursive Problems}
\begin{itemize}
\item $f_{n+1} = f_n + 1$ for $f_0 >= 0$
\item $f_{n+1} = f_n + 2$ for $f_0 >= 1$
\item $f_{n+1} = f_n ^* 2$ for $f_0 = 1$
\item $f_{n+1} = f_n / f_{n - 1}$ for $f_1 = 2, f_0 = 1$ \pause
\item SUPER BONUS: Try to come up with a recursive function that outputs a random number sequence.  \pause
\item SUPER SUPER BONUSE: Create the fill-in algorithm using recursion.
\end{itemize}
\end{frame} 