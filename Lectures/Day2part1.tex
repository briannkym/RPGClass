
\section{Lets Make a Map}

\subsection{Arrays}
\begin{frame}[fragile]{Review \#1}
\begin{semiverbatim}\code{
int i = 30 \% 7; \pause \textbackslash\textbackslash 2
int j = 30 / 5; \pause \textbackslash\textbackslash 6
int k = i * i - j; \pause \textbackslash\textbackslash -2
int h = 0;
if(k <= 0)\{
   h += 4;  \pause \textbackslash\textbackslash 4
   if(k == 0)\{
      h /= 1;
   \} else if (!(k >= -1))\{
      h /= 2;
   \} else \{
      h /= 4;
   \} \pause \textbackslash\textbackslash 2
}\end{semiverbatim}

\end{frame}

\begin{frame}[fragile]{Review \#1}

\begin{semiverbatim}\code{
   if(k == 0)\{
      h /= 1;
   \} else if (!(k >= -1))\{
      h /= 2;
   \} else \{
      h /= 4;
   \} \textbackslash\textbackslash 2
   System.out.print("Zombies eat bra");
   if((false || k == -2)\&\& !(j\textsuperscript{*}h\textsuperscript{*}i > 20))\{
      System.out.print("i");
   \}
   System.out.print("ns and oats.")\pause \textbackslash\textbackslash brans
}\end{semiverbatim}
\end{frame}



\begin{frame}[fragile]{Review \#2}

\begin{semiverbatim}\code{
boolean b; int i = 1; int j = 0;
String s = i + "2" + j + "1" + j; \pause \textbackslash\textbackslash "12020"
double d = Double.parseDouble(s);
double w = (d + i) / d;
if (w <= 0)\{
    b = true;
    s+="hello!"
\} else \{
    b = false;
    s+="goodbye!"
\} \pause \textbackslash\textbackslash "goodbye!"
}\end{semiverbatim}

\end{frame}

\begin{frame}[fragile]{Review \#2}
\begin{semiverbatim}\code{
if (w <= 0)\{
    b = true;
    s+="hello!"
\} else \{
    b = false;
    s+="goodbye!"
\} \textbackslash\textbackslash "goodbye!"
if(!(!s.equals("12010goodbye") || b ) \&\& !b)\{
    System.out.println("Goodbye!")
\} else if (!(!b))\{
    System.out.println("Hello!")
\} \pause \textbackslash\textbackslash no print.
}\end{semiverbatim}

\end{frame}


\begin{frame}{Eclipse}
When writing programs, errors are inevitable, and sometimes frustrating. Syntax errors can be especially frustrating, but luckily there is a tool that highlights such errors:
\begin{itemize}
\item Create a folder on your flash drive called \code{MyProjects}.
\end{itemize}

\begin{center}
Open Eclipse...
\end{center}
\end{frame}

\begin{frame}{From now on we will use Eclipse}
\begin{itemize}
\item Eclipse will ask you to select a workspace, select the \code{MyProjects} folder you just created. You must select this as your workspace everytime you use Eclipse.
\item To start a java project click \code{File$\rightarrow$New$\rightarrow$Project}. Then select \code{Java Project} in the \code{Java} folder and then click \code{Next}.
\item Give your project a name, then click \code{Finish} \pause

\item To create a *.java file, click \code{File$\rightarrow$New$\rightarrow$Class}. Click \code{Finish}. Write a simple program that prints out a line. \pause
\item There are two ways to run your code. Either click ``Run", \code{\emph{the green circle with an arrow in it}}, or \code{Right Click$\rightarrow$Run As$\rightarrow$Java Application}.
\end{itemize}
\end{frame}

\begin{frame}{Output}
\begin{itemize}
\item The output should appear in a terminal in eclipse at the bottom of the screen.
\item By default, errors and problems with your code should also show up here. \pause
\item Other Benefits:
\begin{itemize}
\item Syntax highlighting.
\item Eclipse has code completion. Try typing ``\code{\texttt{System.}}" . You should see a list of options. Click on any one of these options to learn more about it. Press enter to insert an option. \pause
\end{itemize}
\end{itemize}

\end{frame}
\begin{frame}{Some more Benefits}
\begin{itemize}
\item Error correction: Type in the following: ``\code{\texttt{System.out.pintln("Hello");}}. You should see a lightbulb on the left. Click it. The lightbulb offers suggestions as to how to fix your code. Click \code{Change to 'println(..)'}. \pause
\item Auto tabbing: Remove or add white space to your codes (more tabs, less tabs, etc.). Hold \code{\texttt{Ctrl+Shift+F}}. Magic!
\end{itemize}
\end{frame}

\begin{frame}[fragile]{Back to String[] args}
\begin{itemize}
\item What is that \code{\texttt{[]}}?
\item The \code{\texttt{[]}} \emph{declares} that args is an array.
\item An array is a fixed list of variables that all have the same type. \pause
\item For example, when running \code{\texttt{java ArgProgram hi there matey !}}\\
\begin{semiverbatim}\code{
    args[0] = "hi";
    args[1] = "there";
    args[2] = "matey";
    args[3] = "!";
}\end{semiverbatim}
\end{itemize}
\end{frame}

\begin{frame}{Arrays}
\begin{itemize}
\item To declare an array you use:
\begin{center}
\code{\texttt{<type>[] <name>;}}
\end{center}

\item Arrays can have any type. For example:
\begin{center}
\code{\texttt{String[] names;}}
\code{\texttt{double[] vector;}}
\code{\texttt{int[] colors\_list;}}
\code{\texttt{char[] commands;}}
\code{\texttt{boolean[] mask;}}
\end{center}
\end{itemize}
\end{frame}


\begin{frame}{Instantiating Arrays: 2 Ways}
\begin{itemize}
\item Way \#1:
\begin{center}
\code{\texttt{String[] s = \emph{new} String [3];}}\\
\code{\texttt{s[0] = "You see an alligator. What should you do? [Run] [Hide]";}}\\
\code{\texttt{s[1] = "Awww, you were eaten. =(";}}\\
\code{\texttt{s[2] = "You escaped!";}}
\end{center}\pause

\item Way \#2:
\begin{center}
\code{\texttt{int[] colors\_list = \{20, 40, 567433\};}}
\end{center}
\item Similar to Way \#1, Way \#2 creates a new array that can hold 3 elements.\pause
\item Arrays are indexed by integers, thus the following is true:
\begin{center}\code{\texttt{colors\_list[1] == 40}}
\end{center}
\end{itemize}
\end{frame}

\begin{frame}{Some Common Errors}
\code{\texttt{int array[x]}}
\begin{itemize}
\item \code{\texttt{x <= -1}} \pause
\item \code{\texttt{x >= array.length}} \pause
\item \code{\texttt{array = 5 \textbackslash\textbackslash This will cause an error}} \pause
\item \code{\texttt{array[3] = 5 \textbackslash\textbackslash This is right}} \pause
\item \code{\texttt{array[x].length \textbackslash\textbackslash This will cause an error}} \pause
\item \code{\texttt{array.length \textbackslash\textbackslash This is right}}
\end{itemize}
\end{frame}

\begin{frame}{2 Dimensional Arrays}
\begin{itemize}
\item Arrays can have more than one dimension. For example:
\begin{center}\texttt{\code{char[][] map = new map[20][20]}}\end{center}
\item This create a $20 \times 20$ array of characters.
\item We can instantiate the array the same way as a 1D array. For example:
\begin{center}\code{\texttt{map[x][y] = 'S';}}\\
\code{\texttt{map = \{\{'a','b','c'\},\{'d','e','f'\},\{'g','h','i','j'\}\};}}\\ \end{center}
\item To get the number of variables in an array use \code{\texttt{<arrayName>.length}}
\begin{center}\code{\texttt{map.length == 3 \&\& map[2].length == 4}}\\ \end{center}

\end{itemize}

\end{frame}

\begin{frame}[fragile]{Example}
\begin{semiverbatim}\code{
public class BattleShip\{
   public static void main(String[] args)\{
   char[][] map = \{\{' ',' ',' ',' '\},
                \{' ',' ','X',' '\},
                \{' ',' ','X',' '\},
                \{' ',' ','X',' '\}\};
}\end{semiverbatim}
\begin{center}Try printing out different squares in the array.\end{center}
\end{frame}

\subsection{For Loops}

\begin{frame}[fragile]{Complete the Code}
\begin{semiverbatim}\code{
\underline{  1  } \underline{  2  } TaxCalculator\{
    \underline{  3  } \underline{  4  } \underline{  5  }(\underline{  6  }[] \underline{  7  })\{
        if (\underline{  8  }.length \underline{  9  } 1) \underline{  10  }
            \underline{  11  } tax = 0.075;
            double total = \underline{ 12 }.parseDouble(\underline{ 13 });
            total \underline{   14   } 1.0 + tax;
            \underline{   15   }.println("With a rate of " + tax
          + ", the total price is " + total + ".");
        \}
    \}
\}
}\end{semiverbatim}
\end{frame}

\begin{frame}[fragile]{MiniQuiz: Find all the Errors}
\begin{semiverbatim}\code{
Public class \_RollD20\{
    Public statc void main(string[] args)
        int roll = Math.random;
        int d20 = Integer.parseInt(roll * 20)
        system.out.PrintLn('Your roll was ' +
           d + 20);
    \}
}\end{semiverbatim}
\end{frame}

\begin{frame}{Using a For Loop}
\begin{itemize}
\item Now that we know about arrays, we can now use for loops to access many variables at once.
\item For loops look like this:
\begin{center}\texttt{\code{for(int i = 0; i < 5; i++)}}\end{center} \pause
\item We can abstract for loops as:
\begin{center}\texttt{\code{for(<instantiation>; <boolean operation>; <arithmetic operation>)}}\end{center} \pause
\item Both the \texttt{\code{<instantiation>}} and the \texttt{\code{<arithmetic operation>}} are optional.
\end{itemize}
\end{frame}


\begin{frame}{What does it do?}
\begin{itemize}
\item Works like an if statement. If the \texttt{\code{<boolean operation>}} is true then it runs the next line or block of code. \pause
\item Unlike an if statement the for loop keeps running the loop from top to bottom until the \texttt{\code{<boolean operation>}} is false. \pause
\item Before the loop begins \texttt{\code{<instantiation>}} is executed. \pause
\item After each \emph{iteration} of the for loop, the \texttt{\code{<arithmetic operation>}} is executed;
\end{itemize}
\end{frame}


\begin{frame}[fragile]{Printing a Map}
\begin{center}Using the code from before, print out the array.\end{center}
\begin{semiverbatim}\code{
for(int y = 0; y < map.length; y++)\{
  for(int x = 0; x < map[y].length; x++)\{
     System.out.print(map[y][x]);
   \}
   System.out.println();
\}
}\end{semiverbatim}
\end{frame}


\begin{frame}{Other Ways to Use a For Loop}
\begin{itemize}
\item For loops are quite versatile:
\item \texttt{\code{for(int i = 0; i \textless 5; i++)}} \pause
\item \texttt{\code{for(String s = "a"; !s.equals("aaaaa"); s+="a")\{...\}}} \pause
\item \texttt{\code{double d = 0.0; for(d = 1.0; d \textless 5.0; d\textsuperscript{*}=1.2);}} \pause
\item \texttt{\code{boolean b = true for(; b; b=!b)\{...\}}} \pause
\item \texttt{\code{for(int x : colors\_list)\{...\}}} \pause
\item \texttt{\code{for( ; ; )\{...\} \textbackslash\textbackslash runs infinitely!!}}\pause
\end{itemize}
\end{frame}

\begin{frame}{BattleShip}
\begin{center}
Extend the Battleship code such that it works like the game Battleship. Have the computer randomly hide its ships using. Here's a hint:\\

\texttt{\code{int y = (int)(Math.random() * map.length);}}\\

Each turn the player should put in an input, which updates the map with either \texttt{\code{'H'}} for hit or \texttt{\code{'M'}} for miss. \\ \pause
Optionally, extend the code so that you can place your own ships, Give the computer an AI for playing using \texttt{\code{Math.random()}}.
\end{center}
\end{frame}


\begin{frame}[fragile]{Spot the Runtime Errors}
\begin{semiverbatim}\code{
int array = new array[-1];
for(int y = -1; y <= array.length; y <= 1)\{
    array[y] = array[y + 1];
    array = null;
\}
}\end{semiverbatim}
\end{frame}


\begin{frame}[fragile]{Spot the Logical Errors}
\begin{itemize}
\item We want the old array to hold the contents of the array.
\item We want the new array to contain \{9, 7, 5, 3, 6\}
\end{itemize}
\begin{semiverbatim}\code{
int oldarray = array;
int[] array = \{5, 4, 3, 2, 1\};
array = oldarray;
for(int y = 0; y < array.length; y ++)\{
    array[y] += array[(y + 1) \% 4];
\}
}\end{semiverbatim}
\end{frame}

\section{While loops and Switch-Case}
\subsection{}
\begin{frame}[fragile]{While Loops}
\begin{itemize}
\item While loops work just like for loops, except they only use booleans:
\begin{center}\texttt{\code{while(<boolean operation>)}}\end{center}
\item Example:
\begin{semiverbatim}\code{
String s ="";
while(!s.equals("Yes"))\{
    System.out.println("Are you sure?")
    s = input.nextLine();
\}
}\end{semiverbatim}

\end{itemize}
\end{frame}

\begin{frame}[fragile]{Do - While Loops}
\begin{itemize}
\item Do-while loops work the same way as while loops except \emph{they always execute the following block of code at least once.}
\item Example:
\begin{semiverbatim}\code{
String s = input.nextLine();
do \{
    System.out.println("Are you sure?")
    s = input.nextLine();
\}while(!s.equals("Yes"));
}\end{semiverbatim}
\item Challenge: change your Battleship code to use only while or do-while loops.
\end{itemize}

\end{frame}

\begin{frame}[fragile]{Equivalence between For and While Loops}

\begin{semiverbatim}\code{
for(<instantiation>; <boolean operation>;
     <arithmetic operation>)\{
...
\}
}\end{semiverbatim}

\begin{center}$\equiv$\end{center}
\begin{semiverbatim}\code{
<instantiation>;
while(<boolean operation>)\{
...
<arithmetic operation>
\}
}\end{semiverbatim}

\end{frame}

\begin{frame}[fragile]{Switch-Case}
\begin{itemize}
\item Often in programming we need to check several different cases. For example:
\begin{semiverbatim}\code{
int i = 5;
if (i == 0)\{
...
\} else if (i == 1)\{
...
\} else if (i == 2)\{
...
\} else ....
}\end{semiverbatim}
\item This takes a long time. Instead, we can use Switch-Case statements.
\end{itemize}
\end{frame}

\begin{frame}[fragile]{Quick Experiment!}
\begin{semiverbatim}\code{
int i = \underline{   };
switch(i)\{
   case 0:
   System.out.println("0!");   break;
   case 1:
   System.out.println("1!");   break;
   case 2:
   System.out.println("2!");
   ...
}\end{semiverbatim}
\end{frame}

\begin{frame}[fragile]{Quick Experiment!}
\begin{semiverbatim}\code{
   case 3:
   System.out.println("3!");
   case 4:
   System.out.println("4!");   break;
   default:
   System.out.println("Other");
\}
}\end{semiverbatim}
\end{frame}

\begin{frame}[fragile]{Why use Switch-Case?}
\begin{itemize}
\item Switch case simplifies code, and gets rid of if-else chains. \pause
\item It is much faster!
\begin{center}
Convert the experiment code to a chain of if-else statements. Use \texttt{\code{System.nanoTime()}} to measure how long the code takes. Example: \pause
\end{center}
\begin{semiverbatim}\code{
long timeBefore = System.nanoTime();
\textbackslash\textbackslash Your code here.
long timeAfter = System.nanoTime();
long totalTime = timeAfter - timeBefore;
}\end{semiverbatim} \pause
\item Why use if-statements? \pause \emph{You need if statements for any non ``discrete" values, like doubles or strings.}
\end{itemize}
\end{frame}

\begin{frame}{Pop Quiz!}
\begin{center}
What is the difference between Declaration and Instantiation?\\  \pause

Can an array be declared and have all of its variables instantiated in one line?\\  \pause

What does \texttt{\code{System.nanoTime()}} do?\\  \pause

How do you make an infinite for loop?\\  \pause

Are while loops and for loops equivalent?\\  \pause

Give an example of a scenario where we can't use switch-case statements.

\end{center}
\end{frame}

\begin{frame}{Implement a 2D map based RPG}

\begin{center}
From now until Lunch, try to implement a 2D RPG with a Player, 'P', and a monster 'M'. The player should battle the monster if the player is on the same square.
\end{center}
\end{frame}
